\documentclass[fyma,pdf,final]{prosper}

\usepackage[czech]{babel}
\usepackage[utf8]{inputenc}
\usepackage[T1]{fontenc}
\usepackage{picture}
\usepackage{graphics}
\usepackage{listings}


\slideCaption{\textit{WWW}}
\DefaultTransition{Wipe}

\begin{document}
	\title{WWW}
	\subtitle{Vývoj webových stránek}
	\author{Marek Schauer}
	\email{xschau00@stud.fit.vutbr.cz}
	\institution{Vysoké učení technické v~Brně\\Fakulta informačních technologií}
	\maketitle

	\begin{slide}{Proč se naučit vyvýjet weby?}
		\begin{itemize}
			\item Po tvorbě webových stránek je neustálá poptávka - když se weby naučíte dělat dobře, o obživu můžete mít postaráno
			\item Naučit se tvořit weby dobře nemusí být nic těžého
			\item Kdy radši vývoj webu nechat na odborníky?
				\begin{itemize}
					\item Když Vaše firma potřebuje profesionální web - zvládnutí všech webových technologií by Vás jako majitele firmy mohlo stát spoustu času
				\end{itemize}

		\end{itemize}
	\end{slide}

	\begin{slide}{Webový vývojář včera a dnes}
		\begin{itemize}
			\item V minulosti nebylo výjimkou, že celý web dělal 1 člověk.
			\item Dnes je však situace poměrně odlišná. Počas vývoje internetu a webových stránek se vytvořily 2 hlavní větvy webového vývoje, kterým se může vývojář vydat:
				\begin{itemize}
					\item Frontend developer - implementuje uživatelské části webu
					\item Backend developer - zajišťuje správný chod webové aplikace na pozadí, která zvyčejně beží na serveru
				\end{itemize}
			\item Tímto dělením se však výčet profesí spojených s webovým vývojem nekončí. Svoje uplatnění mají také:
				\begin{itemize}
					\item Grafik - navrhuje dizajn webu
					\item Copywriter - píše poutavé texty pro webovou prezentaci
					\item Webový konzultant - definuje, co přesně zákazník potřebuje a pomáhá mu uvědomit si, co je pro něho při vytváření webu důležité
				\end{itemize}
		\end{itemize}
	\end{slide}

	\begin{slide}{Základní technologie spojené s tvorbou webu}
		Základem všech webů jsou tyto jazyky:
		\begin{itemize}
			\item HTML
				\begin{itemize}
					\item značkovací jazyk, používa se pro obsahovou stránku webové prezentace
					\item nejnovější verze je HTML 5
				\end{itemize}
			\item CSS
				\begin{itemize}
					\item jazyk spojený s vizuální částí webu
					\item nejnovější verze CSS je 3

				\end{itemize}
		\end{itemize}
		Základem každého webového vývojáře musí být perfektní znalost HTML a CSS.
	\end{slide}

	\begin{slide}{Programovací jazyky}
	S vývojem webových stránek se pojí také programovací jazyky.
		\begin{itemize}
			\item JavaScript
				\begin{itemize}
					\item programovací jazyk, který na základnějších webových prezentacích dopomáhá spíše k vizuální stránce
					\item při rozsáhlejších webových projektech může hrát i významnější funkční roli
					\item vykonává se v prohlížeči uživetele
				\end{itemize}
			\item PHP
				\begin{itemize}
					\item programovací jazyk, který slouží k dynamickému generování webových stránek
					\item na rozdil od JavaScriptu se vykonává na serveru, kde je webová stránka uložena
				\end{itemize}
		\end{itemize}
	\end{slide}


	\begin{slide}{HTML}
		\begin{itemize}
			\item HTML je zkratka pro Hypertext Markup Language
			\item je složen ze značek v ostrých závorkách, které určují smysl a vzhled dokumentu
				\begin{itemize}
					\item tímto značkám se říká tagy
				\end{itemize}
			\item Každá webová stránka je tvořená HTML dokumentem, který má nasledující nebo podobnou strukturu:
		\end{itemize}
\begin{lstlisting}[frame=single]  % Start your code-block
<!DOCTYPE html>
<html>
   <head>
      <title></title>
   </head>
   <body>

   </body>
</html>
\end{lstlisting}
	\end{slide}

	\begin{slide}{CSS}
		\begin{itemize}
			\item CSS je zkratka pro Cascading Stylesheets
			\item jde o stylový předpis, který definuje, jak webová stránka vypadá
			\item důvodem vzniku CSS je oddělení vizuální a obsahové části webu
				\begin{itemize}
					\item v minulosti se obě tyto části psali v HTML
				\end{itemize}
			\item CSS kód má následující syntax:
\begin{lstlisting}[frame=single]  % Start your code-block

selektor {
   vlastnost: hodnota;
}
\end{lstlisting}

		\end{itemize}
	\end{slide}
		
	\begin{slide}{V čem se weby píšou?}
		Webové stránky se píšou v textových editorech, zde je několik příkladů:
		\begin{itemize}
			\item NotePad
			\item Sublime Text
				\begin{itemize}
					\item výhodou je rychlost, pružnost, Package Controller
				\end{itemize}
			\item Adobe Dreamweaver
				\begin{itemize}
					\item obsahuje i WYSIWYG
				\end{itemize}
		\end{itemize}
	\end{slide}
	

	\begin{slide}{Responzivita}
		\begin{itemize}
			\item přizpůsobení zobrazení webu všem platformám
			\item implementuje se pomocí tzv. mediaqueries
				\begin{itemize}
					\item založeno na definici CSS vlastností pro vymezené rozlišení
				\end{itemize}
			\item ukázka stylování pomocí mediaqueries:
\begin{lstlisting}[frame=single]  % Start your code-block

@media only screen and (max-width: 500px) {
   body {
       background-color: lightblue;
   }
}
\end{lstlisting}
		\end{itemize}
	\end{slide}
	\begin{slide}{Ukázka responzivity}
		\begin{figure}[h]
			\begin{center}
		        \includegraphics{resp.eps}
			    \caption{Ukázka responzivity}
			    \label{pic:responsivity}
			\end{center}
		\end{figure}
	\end{slide}
		
\end{document}